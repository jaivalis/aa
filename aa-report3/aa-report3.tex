\documentclass[a4paper,10pt]{article}
\usepackage{array}
\usepackage{tabularx}
\usepackage{graphicx}
\usepackage{algorithm}
\usepackage{algorithmic}
\usepackage{pgfplotstable}
\usepackage{pgfplots}
\usepackage{filecontents}
\usepackage{amsmath}
\usepackage{float}



\title{
	\textbf{Autonomous Agents Assignment 3}
}

\author{Tobias Stahl \\ 10528199 \and Spyros Michaelides \\ 10523316 \and Ioannis Giounous Aivalis \\ 10524851 \and Francesco Stablum \\ 6200982}




\begin{document}

\maketitle


\section{Introduction}

%--A general framework for explaining your results/experiments is to follow these points : 

%HYPOTHESIS
%--1) why? "in order to test..." (questions, and preferably hypotheses with explanation) --> Hypothesis
%--2) is there anything to mention about the implementation/machinery the reader should know? (e.g. "these experiments were performed on a [insert machine specs]" when presenting runtimes) - 

%RESULTS

%Interpretation
%--3) what does it show (and is that what you expected, and why)

%Findings
%--4) take home message (what do you want the reader to remember, e.g. "Therefore, by reducing the state space, we have gained several orders of magnitude in runtime.")


\section{New Environment}
%intro, summary


\subsection{Experiment}



\subsubsection{Hypothesis}


\subsection{Results}
%Interpretation


%Findings







\section{minimax-Q}
%intro, summary


\subsection{Experiment}



\subsubsection{Hypothesis}


\subsection{Results}
%Interpretation


%Findings





\section{Discussion}



\section{Conclusion}


\begin{thebibliography}{9}

\bibitem{sutton}
  Richard S. Sutton and Andrew G. Barto ,
  \emph{Reinforcement Learning: An Introduction}.
  The MIT Press, Cambridge, Massachusetts

\end{thebibliography}

\end{document}
